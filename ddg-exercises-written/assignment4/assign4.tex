\documentclass{article}

\usepackage{blindtext}
\usepackage{graphicx}
\usepackage{wrapfig}
\usepackage[skip=1ex]{caption}
\usepackage{subcaption}
\usepackage{mdframed}
\usepackage{amsmath}
\usepackage{amsfonts}
\usepackage{amssymb}
\usepackage{amstext}
\usepackage{cancel}
\usepackage{enumitem}
\usepackage[english]{babel}
\usepackage{helvet}
\usepackage{microtype}
\usepackage[pdftex]{hyperref}
\usepackage{float}
\usepackage{nicematrix}
\usepackage{xcolor}
\usepackage{tikz}
\usepackage{geometry}
\geometry{
    a4paper,
    left=2cm,
    right=2cm,
    top=1cm,
    bottom=1cm
}

\special{papersize=8.5in,11in}
\setlength{\pdfpageheight}{\paperheight}
\setlength{\pdfpagewidth}{\paperwidth}

% Macros

% Make inline frac bigger
\newcommand\ifrac[2]{{\displaystyle\frac{#1}{#2}}}

% Aliases
\def\wstar{\overset{*}{\rightharpoonup}}
\def\grad{\nabla}
\def\lap{\Delta}
\def\nt{\notag}
\def\dt{\partial_t}
\def\hal{\ifrac{1}{2}}
\def\ep{\varepsilon}
\def\cK{\mathcal{K}}
\def\cA{\mathcal{A}}
\def\cS{\mathcal{S}}
\def\cV{\mathcal{V}}
\def\cJ{\mathcal{J}}
\def\Q{\mathbb{Q}}
\def\R{\mathbb{R}}
\def\R{\mathbb{R}}
\def\C{\mathbb{C}}
\def\la{\langle}
\def\ra{\rangle}
\def\ll{\langle\langle}
\def\rr{\rangle\rangle}

% Custom operators
\DeclareMathOperator{\Err}{Err}
\DeclareMathOperator{\re}{Re}
\DeclareMathOperator{\im}{Im}





\begin{document}



\title{Written Assignment 4}

\author{Niraj Venkat}

\date{}

\maketitle

\vspace{.8cm}
\boxed{\text{Exercise} \quad 1}\\\\

Important to note that we can only prove this for a 1-form $\alpha$ in $\R^2$. In $\R^n$ the Hodge star of a 
$k$-form is an $(n-k)$-form. We now need some way of relating the operators Hodge star $\star$ and complex structure $\cJ$.\\

Adopting the convention $\star\alpha(X) = \alpha(\cJ X)$:

\begin{align*}
    \star\alpha(X) \wedge \alpha(X, \cJ X) &= \star\alpha(X)\alpha(\cJ X) - \star\alpha(\cJ X)\alpha(X) \\
        &= \alpha(\cJ X)\alpha(\cJ X) - \alpha(\cJ \cJ X)\alpha(X) \tag*{$\cJ^2$ = -id} \\
        &= \alpha(\cJ X)\alpha(\cJ X) + \alpha(X)\alpha(X) \\
        &\ge 0
\end{align*}

$X$ and $\cJ X$ form an orthogonal basis for the tangent space $T_pM$. This means for any two 
real valued 1-forms $\alpha, \beta \ge 0$ we claim that 
$\langle \langle \alpha, \beta \rangle \rangle = \int_M \star\alpha \wedge \beta$ is positive definite.\\

With the equal and opposite convention $\star\alpha(X) = - \alpha(\cJ X)$:

\begin{align*}
    \alpha(X) \wedge \star\alpha(X, \cJ X) &= \alpha(X)\star\alpha(\cJ X) - \alpha(\cJ X)\star\alpha(X) \\
        &= -\alpha(X)\alpha(\cJ \cJ X) + \alpha(\cJ X)\alpha(\cJ X) \tag*{$\cJ^2$ = -id} \\
        &= \alpha(X)\alpha(X) + \alpha(\cJ X)\alpha(\cJ X) \\
        &\ge 0
\end{align*}
In this case we claim that 
$\langle \langle \alpha, \beta \rangle \rangle = \int_M \alpha \wedge \star\beta$ is positive definite.


\vspace{1.8cm}
\boxed{\text{Exercise} \quad 2}\\\\


\begin{align*}
    (\star \star \alpha) \wedge (\star \alpha) (X, \cJ X) &= \star \star \alpha(X) \star \alpha(\cJ X) - \star \star \alpha(\cJ X) \star \alpha(X) \\
        &= \star \alpha(\cJ X) \star \alpha(\cJ X) + \star \alpha(X) \star \alpha(X) \\
        &= \alpha(\cJ X) \alpha(\cJ X) + \alpha(X) \alpha(X) \\
        &= \star \alpha \wedge \alpha (X, \cJ X)
\end{align*}

Therefore, 

\begin{align*}
    ||\star \alpha|| &= \sqrt{\langle \langle \star \alpha, \star \alpha \rangle \rangle} \\
        &= \sqrt{\int_M (\star \star \alpha) \wedge (\star \alpha)} \\
        &= \sqrt{\int_M \star \alpha \wedge \alpha} \\
        &= \sqrt{\langle \langle \alpha, \alpha \rangle \rangle} = ||\alpha||
\end{align*}

The geometric intuition here is that in $\R^2$, the Hodge star of a 1-form is just a $90^\circ$ rotation,
and rotations preserve length.


\vspace{1.8cm}
\boxed{\text{Exercise} \quad 3}\\\\


Let $u = a + i b$ and $v = c + i d$, then

\begin{align*}
    \bar{u} v &= (a - i b) (c + i d) \\
        &= ac + i ad - i bc + bd \\
        &= (ac + bd) + i (ad - bc) \\
        &= u \cdot v + i (u \times v)
\end{align*}


\vspace{1.8cm}
\boxed{\text{Exercise} \quad 4}\\\\


We show that inner product $\langle. \,, .\rangle$ in $\C$ is Hermitian:

\begin{align*}
    \langle u, v \rangle &= u \cdot v + i (u \times v) \\
        &= v \cdot u - i (v \times u) \\
        &= \overline{v \cdot u - i (v \times u)} \\
        &= \overline{\langle v, u \rangle}
\end{align*}

Next we show that inner product is positive definite for $u \ne 0$:

\begin{align*}
    \langle u, u \rangle &= u \cdot u + i (u \times u) \\
        &= u \cdot u + 0 \\
        &> 0
\end{align*}


\vspace{1.8cm}
\boxed{\text{Exercise} \quad 5}\\\\


\begin{align*}
    \star \bar{\alpha} \wedge \alpha (X, \cJ X) &= \star \bar{\alpha}(X) \alpha(\cJ X) - \star \bar{\alpha}(\cJ X) \alpha(X) \\
        &= \bar{\alpha}(\cJ X) \alpha(\cJ X) + \bar{\alpha}(X) \alpha(X) \\
        &= \langle \alpha(\cJ X), \alpha(\cJ X) \rangle + \langle \alpha(X), \alpha(X) \rangle \\
        &\ge 0
\end{align*}

\begin{align*}
    \star \bar{\alpha} \wedge \beta (X, \cJ X) &= \star \bar{\alpha}(X) \beta(\cJ X) - \star \bar{\alpha}(\cJ X) \beta(X) \\
        &= \bar{\alpha}(\cJ X) \beta(\cJ X) + \bar{\alpha}(X) \beta(X) \\
        &= \langle \alpha(\cJ X), \beta(\cJ X) \rangle + \langle \alpha(X), \beta(X) \rangle
\end{align*}

\begin{align*}
    \star \bar{\beta} \wedge \alpha (X, \cJ X) &= \langle \beta(\cJ X), \alpha(\cJ X) \rangle + \langle \beta(X), \alpha(X) \rangle \\
        &= \overline{\langle \alpha(\cJ X), \beta(\cJ X) \rangle} + \overline{\langle \alpha(X), \beta(X) \rangle} \\
        &= \overline{\star \bar{\alpha} \wedge \beta (X, \cJ X)}
\end{align*}

Using the identity $z + \bar{z} = \re z$, $\langle \langle \alpha, \beta \rangle \rangle = \re \int_M \star \bar{\alpha} \wedge \beta$
is Hermitian and positive definite.


\vspace{1.8cm}
\boxed{\text{Exercise} \quad 6}\\\\


Because $u,v$ are 0-forms, $du,dv$ are 1-forms.\\
We will use this in the product rule for exterior derivatives when considering Stokes' theorem on the manifold $M$.\\
Assuming that $du,dv$ are zero at the boundary $\partial M$:

$$
    \re \int_{\partial M} v \star \overline{du} = 0
$$

\begin{align*}
    0 &=\re \int_{\partial M} v \star \overline{du} \\
        &= \re \int_M d(v \star \overline{du}) \\
        &= \re \int_M dv \wedge \star \overline{du} + \re \int_M (-1)^0 v \wedge d \star \overline{du} \\
        &= -\re \int_M \star \overline{du} \wedge dv + \re \int_M v \wedge d \star \overline{du} \\
    \implies \re \int_M \star \overline{du} \wedge dv &= \re \int_M v \wedge d \star \overline{du}
\end{align*}
where we used Stokes' theorem to convert $\int_{\partial M} \cdots \rightarrow \int_M d(\cdots)$\\

From our previous result for the inner product of complex 1-forms, we have:
\begin{align*}
    \langle \langle du, dv \rangle \rangle &= \re \int_M \star \overline{du} \wedge dv  \\
        &= \re \int_M v \wedge d \star \overline{du} \\
        &= -\re \int_M d \star \overline{du} \wedge v \\
        &= \re \int_M (\star \star) d \star \overline{du} \wedge v \tag*{$\star^2 = $ -id when $n=2, k=1$} \\
        &= \re \int_M \star (\star d \star d) \bar{u} \wedge v \tag*{$\overline{du} = d\bar{u}$} \\
        &= \re \int_M \star \lap\bar{u} \wedge v \tag*{$\lap = \star d \star d$} \\
        &= \langle \langle \lap u, v \rangle \rangle
\end{align*}


\vspace{1.8cm}
\boxed{\text{Exercise} \quad 7}\\\\


Area on a Riemann surface (topological disk $M$) can be expressed in terms of exterior calculus as $X \times \cJ X$.\\
A conformal parameterization $z : M \rightarrow \C$ must satisfy the Cauchy-Riemann equation: 
$$ dz(\cJ X) = idz(X) \implies \star dz = idz $$
Total signed area of the region $z(M)$:
\begin{align*}
    \cA(z) &= \int_M dz \times d\bar{z} \\
        &= \hal \int_M \star (dz \wedge d\bar{z}) \tag*{Signed vector area formula} \\
        &= \frac{i}{2} \int_M dz \wedge d\bar{z} \tag*{using Cauchy-Riemann} \\
        &= -\frac{i}{2} \int_M d\bar{z} \wedge dz
\end{align*}


\vspace{1.8cm}
\boxed{\text{Exercise} \quad 8}\\\\


Using \href{https://projecteuclid.org/ebooks/proceedings-of-the-centre-for-mathematics-and-its-applications/Theoretical%20and%20Numerical%20Aspects%20of%20Geometric%20Variational%20Problems/chapter/Computing%20conformal%20maps%20and%20minimal%20surfaces/pcma/1416323558}{this paper}
by John E. Hutchinson as reference, whos argument works for the smooth case.\\

The conformal energy $E_C(z)$ is defined as the failure of the map $z$ to be conformal, and as such, satisfying the discrete
Cauchy-Riemann equations.\\

We prove that $E_C(z)$ is the difference of Dirichlet energy $E_D(z) = \hal \ll \lap z, z \rr$ and total signed area $\cA$:
\begin{align*}
    E_C(z) &= \frac14 || \star dz - idz ||^2 \\
        &=  \frac14 \ll \star dz - idz, \star dz - idz \rr \tag*{Inner product is distributive} \\
        &=  \frac14 \Big( \ll \star dz, \star dz \rr - \ll \star dz, idz \rr - \ll idz, \star dz \rr + \ll idz, idz \rr \Big) \\
        &=  \frac14 \Big( 2 \ll \star dz, \star dz \rr - 2i\ll \star dz, dz \rr \Big) \tag*{Inner product is homogeneous} \\
        &=  \hal \ll \star dz, \star dz \rr - \frac{i}{2} \ll \star dz, dz \rr \\
        &=  \hal \ll \lap z, z \rr - \frac{i}{2} \int_M (\star \star d\bar{z}) \wedge dz \\
        &=  \hal \ll \lap z, z \rr + \frac{i}{2} \int_M d\bar{z} \wedge dz \\
        &=  E_D(z) - \cA(z)
\end{align*}


\vspace{1.8cm}
\boxed{\text{Exercise} \quad 9}\\\\


Using \href{https://www.cs.cmu.edu/~kmcrane/Projects/Other/ConformalGeometryOfSimplicialSurfaces.pdf}{this paper}
by Keenan Crane as reference, which reminded me of our previous result for the complex inner product which we use below.\\

We had proved in a past assignment that signed area of a polygon is the sum of cross products of neighbouring oriented 
boundary edges, which sums up $n$ terms for an $n$-gon.\\

Using this result for $n$-gons to find the area of piecewise linear region $z(M)$:

\begin{align*}
    \cA(z) &= \hal \sum_{ij \in \partial E_\partial} z_i \times z_j \\
        &= \hal \sum_{ij \in \partial E_\partial} \im (\bar{z_i} z_j) \tag*{$z_i \times z_j := \im (\bar{z_i} z_j)$} \\
        &= \hal \sum_{ij \in \partial E_\partial} \frac{\bar{z_i}z_j - \bar{z_j}z_i}{2i} \\
        &= \hal \sum_{ij \in \partial E_\partial} \frac{\bar{z_i}z_j - \bar{z_j}z_i}{2i} \\
        &= -\frac{i}{4} \sum_{ij \in \partial E_\partial} \bar{z_i}z_j - \bar{z_j}z_i
\end{align*}


\pagebreak
\boxed{\text{Exercise} \quad 10}\\\\


From the \href{https://geometrycollective.github.io/boundary-first-flattening/}{Boundary First Flattening paper} by Sawhney et. al.\\

Conformal maps can also be expressed as pairs of conjugate harmonic functions.
A real function $a : M \rightarrow \R$ is harmonic if it sits in the kernel of the Laplace-Beltrami operator $\lap$,
i.e., it solves the Laplace equation $\lap a = 0$.\\

Suppose we express a holomorphic map as $z = a + ib$ for a pair of coordinate functions $a, b : M \rightarrow \R$.
Then (by Cauchy-Riemann)
$$\cJ \grad a = \grad b$$
i.e., the gradients $\grad$ of the two coordinates are orthogonal and have equal magnitude.\\

Since a quarter-rotation of a gradient field is divergence-free, we have
$$\lap a = \grad \cdot \grad a = -\grad \cdot (\cJ \grad b) = 0$$
and similarly, $\lap b = 0$. In other words, the two real components of a holomorphic function
are both harmonic — we say that $a$ and $b$ form a \emph{conjugate harmonic} pair.


\vspace{1.8cm}
\boxed{\text{Exercise} \quad 11}\\\\


If a harmonic function $\varphi : M \rightarrow \C$ is real valued then we do not satisfy Cauchy-Riemann conditions any more,
so $\varphi$ will not be holomorphic/conformal.\\
Geometrically we can interpret $\varphi$ as mapping vectors in $M$ to the real line in $\C$, so the angles in $M$ 
aren't being preserved. \\

A more mathematical treatment comes from \href{https://doi.org/10.1016/0040-9383(76)90042-2}{Eells \& Wood}, which says that:
\begin{mdframed}
    If $f : M_1 \rightarrow M_2$ is a harmonic map and 
    $$\chi(M_1) + |\deg(f)\chi(M_2)| > 0,$$
    then $f$ is either holomorphic or anti-holomorphic (where $\chi(M)$ is the Euler characteristic of $M$, and $\deg(f)$
    is the topological degree of $f$).
\end{mdframed}
We know that $\chi(M) = 1$ because $M$ is a topological disk but both topological degree of $\varphi$ and Euler characteristic 
of the image of $\varphi$ are undefined, so $\varphi$ cannot be holomorphic or anti-holomorphic.


\vspace{1.8cm}
\boxed{\text{Exercise} \quad 12}\\\\


This proof is adapted from \href{https://linear.axler.net/}{Linear Algebra Done Right} by Sheldon Axler:\\

Suppose $A = A^*$ is a self-adjoint operator on a complex inner-product space $V$, with Hermitian inner product $\ll.\,,.\rr$.
Let $\lambda$ be an eigenvalue of $A$, and let $v$ be a nonzero vector in $V$ such that $Av=\lambda v$. Then
$$
    \lambda||v||^2 = \ll\lambda v, v \rr = \ll A v, v \rr = \ll v, A^* v \rr = \ll v, Av \rr = \ll v, \lambda v \rr = \bar{\lambda}||v||^2
$$
So $\lambda = \bar{\lambda}$, which means all the eigenvalues of a self-adjoint operator are real.


\pagebreak
\boxed{\text{Exercise} \quad 13}\\\\


From Axler:\\

An operator on an inner product space is called normal if it commutes with its adjoint, i.e., $AA^* = A^*A$.\\
If $A$ is normal so is $A - \lambda I$ which gives us:
$$
    0 = ||(A - \lambda I)v|| = ||(A - \lambda I)^* v|| = ||(A^* - \bar{\lambda} I)v||
$$

Suppose $\lambda_i, \lambda_j$ are distinct eigenvalues of $A$ with corresponding eigenfunctions $e_i, e_j$.
We have $Ae_i = \lambda_i e_i$ and $Ae_j = \lambda_j e_j$. Furthermore, $A^*e_j = \bar{\lambda_j}e_j$. Thus:
\begin{align*}
    (\lambda_i -  \lambda_j) \ll e_i, e_j \rr &= \ll \lambda_i e_i, e_j \rr  - \ll e_i, \bar{\lambda_j}e_j \rr \\
        &= \ll A e_i, e_j \rr - \ll e_i, A^* e_j \rr \\
        &= 0
\end{align*}
Because $\lambda_i \neq \lambda_j$ this implies $\ll A e_i, e_j \rr = 0$, i.e., $e_i$ and $e_j$ are orthogonal.\\
Every self-adjoint operator should be normal. If $A$ is self-adjoint: $A^* = A$ and $\bar{\lambda} = \lambda$. \\
This proves that eigenvectors of a self-adjoint operator with unique eigenvalues must be orthogonal.


\vspace{1.8cm}
\boxed{\text{Exercise} \quad 14}\\\\


Below is a restatement of Example 1.27 from \href{https://people.csail.mit.edu/jsolomon/}{Numerical Algorithms by Justin Solomon}:\\

Our goal is to minimize $x^TAx$ for a PSD symmetric matrix $A$ subject to our contraint $||x||^2 = 1$,
which is equivalent to $\ll x,\,x \rr = ||x||^2 = 1$.\\

Without the constraint the function is minimized at $x = 0$. We define the Lagrange multiplier function:
$$
    \Lambda(x, \lambda) = x^TAx - \lambda(||x||^2 - 1) = x^TAx - \lambda(x^Tx - 1)
$$
Differentiating w.r.t $x$, we find $0 = \grad_x \Lambda = 2Ax - 2\lambda x$. In other words, critical points of $x$ are exactly
the eigenvectors of the matrix $A$:
$$
    Ax = \lambda x, \quad \text{with} \quad ||x||^2 = 1
$$
At these critical points, we can evaluate the objective function as $x^TAx = x^T\lambda x = \lambda||x||^2 = \lambda$.\\
Hence, the minimizer of $x^TAx$ subject to $||x||^2 = 1$ is the eigenvector $x$ with minimum eigenvalue $\lambda$.


\vspace{1.8cm}
\boxed{\text{Exercise} \quad 15}\\\\

Section 6.3.1 from Solomon:\\

Assume that $A \in \R^{n \times n}$ is non-defective and nonzero with all real eigenvalues,
e.g., $A$ is symmetric. By definition, $A$ has a full set of eigenvectors $x_1,\dots,x_n \in \R^n$; 
we sort them such that their corresponding eigenvalues satisfy $|\lambda_1| \ge |\lambda_2| \ge \dots \ge |\lambda_n|$.\\

Take an arbitrary vector $v \in \R^n$. Since the eigenvectors of $A$ span $\R^n$, we can write
$v$ in the $x_i$ basis as $v = c_1x_1 + \dots + c_nx_n$. Applying $A$ to both sides:

\begin{align*}
    Av &= c_1Ax_1 + \dots + c_nAx_n \\
        &= c_1\lambda_1x_1 + \dots + c_n\lambda_nx_n \tag*{since $Ax_i = \lambda x_i$} \\
        &= \lambda_1 \Big( c_1x_1 + c_2\frac{\lambda_2}{\lambda_1}x_2 + \dots + c_n\frac{\lambda_n}{\lambda_1}x_n \Big) \\
    A^2v &= \lambda_1^2 \Big(c_1x_1 + c_2\Big(\frac{\lambda_2}{\lambda_1}\Big)^2x_2 + \dots + c_n\Big(\frac{\lambda_n}{\lambda_1}\Big)^2x_n \Big) \\
    \vdots \\
    A^kv &= \lambda_1^k \Big(c_1x_1 + c_2\Big(\frac{\lambda_2}{\lambda_1}\Big)^kx_2 + \dots + c_n\Big(\frac{\lambda_n}{\lambda_1}\Big)^kx_n \Big)
\end{align*}

As $k \rightarrow \infty$, the ratio $\Big(\ifrac{\lambda_i}{\lambda_1}\Big)^k \rightarrow 0$ unless $\lambda_i = \pm \lambda_1$,
since $\lambda_1$ has the largest magnitude of any eigenvalue by construction. If $x$ is the projection of $v$ onto the space
of eigenvectors with eigenvalues $\lambda_1$, then -- at least when the absolute values $|\lambda_i|$ are unique --
as $k \rightarrow \infty$ the following approximation begins to dominate: $A^kv \approx \lambda_1x$.\\

By composing this algorithm, called \emph{power iteration}, we produce vectors $v_k$ more and more parallel
to the desired $x_1$ as $k \rightarrow \infty$. Power iteration tends to be much more reliable than standard iterative
methods like \href{https://www.cs.cmu.edu/~quake-papers/painless-conjugate-gradient.pdf}{conjugate gradient}.
Iterative solvers which may be very slow to converge for, e.g., poor triangulations,
are much better suited for very large meshes, where matrix factors may not be able to fit into memory. 


\vspace{1.8cm}
\boxed{\text{Exercise} \quad 16}\\\\


Because $A$ has an inverse:
$$
    Ax = \lambda x \implies x = \lambda A^{-1}x \implies A^{-1}x = \ifrac{1}{\lambda}x
$$
So we retain the same eigenvectors, but now $\frac{1}{\lambda}$ is an eigenvalue of $A^{-1}$.
We can visualize this as the opposite of stretching/squishing
that happened to the eigenvectors of $A$, because we now scale them by $\frac{1}{\lambda}$ instead.



























































\end{document}
